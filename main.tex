\documentclass{article}
\usepackage[utf8]{inputenc}
\usepackage[T1]{fontenc}
\usepackage[ngerman, main=english]{babel}
\usepackage[left=2cm, right=2cm]{geometry}
\usepackage[onehalfspacing]{setspace}
\usepackage{amsmath}
\usepackage{graphicx}

\title{Linux crashcourse}
\author{Schmidt Tristan }
\date{Juni 2021}

\begin{document}

\maketitle
\clearpage
\tableofcontents
\clearpage
\begin{abstract}
    In this document one will learn the basics of the Linux operating system. Furthermore, my paper will shorty explain the rich history of the GNU project and Linux.
\end{abstract}
\section{Introduction}
GNU/Linux is a free project. Free as in freedom and not price. This means that the kernel is completely open source and can be altered by anyone. This has huge implications on security, on the users privacy and on the freedom of the user to customize the operating system to his liking. So there may be benefits gained by switching from Windows/MacOs to Linux.     

\end{document}
